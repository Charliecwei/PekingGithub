\documentclass[11pt]{article}
\usepackage{amsmath,amssymb}
\usepackage{bm}
\usepackage{xcolor}
\usepackage{ctex}
\usepackage{CJK}
\usepackage{float}
\usepackage{amsthm}
\usepackage{geometry}
\geometry{left=2.0cm,right=2.0cm,top=2.5cm,bottom=2.5cm}
\usepackage{algorithmic}
\usepackage{algorithm}           
\newtheorem{myDef}{Definition}[subsection]
\newtheorem{myTheo}{定理}[subsection]
\newtheorem{mylem}{Lemma}[subsection]
\newtheorem{mycor}{推论}[subsection]
\newtheorem{myproof}{证明}[subsection]


\usepackage{geometry} 
\usepackage{indentfirst}
\usepackage{amsthm}
\usepackage{amsmath}
\usepackage{amssymb}
\usepackage{setspace}
\usepackage{geometry}
\usepackage{algorithm}
\usepackage{algorithmic}
\usepackage{multirow}
\usepackage{blkarray}
\usepackage{tikz}

%%%
\newtheorem{theorem}{定理}[subsection]
\floatname{algorithm}{算法}


\usepackage{geometry}      
\geometry{left=2.5cm,right=2.5cm,top=3cm,bottom=3.5cm}    
\usepackage{indentfirst}
\usepackage{CJK}
\usepackage{amsthm}
\usepackage{amsmath}
\usepackage{amssymb}
\usepackage{graphicx}
\usepackage{listings}
\usepackage{setspace}
\usepackage{geometry}
\usepackage{multirow}
\usepackage{xcolor}
\usepackage{cite}
\usepackage{fancyhdr}
\usepackage{ctex}
\usepackage{blkarray}


\newtheorem{lemma}{引理}[subsection]
\newtheorem{inference}{推论}[subsection]

\lstset{ 
    basicstyle=\tt,
    rulesepcolor=\color{red!20!green!20!blue!20},
    escapeinside=``,
    xleftmargin=2em,xrightmargin=2em, aboveskip=1em,
    framexleftmargin=1.5mm,
    frame=shadowbox,
    backgroundcolor=\color[RGB]{245,245,244},
    keywordstyle=\color{blue}\bfseries,
    identifierstyle=\bf,
    numberstyle=\color[RGB]{0,192,192},
    commentstyle=\it\color[RGB]{96,96,96},
    stringstyle=\rmfamily\slshape\color[RGB]{128,0,0},
    showstringspaces=false
}



\lstdefinestyle{C}{
 language={[ANSI]C},
 numbers=left,
 numberstyle=\tiny,
 basicstyle=\small\ttfamily,
 stringstyle=\color{purple},
 keywordstyle=\color{blue}\bfseries,
 commentstyle=\color{olive},
 directivestyle=\color{blue},
 frame=shadowbox,
 %framerule=0pt,
 %backgroundcolor=\color{pink},
 rulesepcolor=\color{red!20!green!20!blue!20}
 %rulesepcolor=\color{brown}
 %xleftmargin=2em,xrightmargin=2em,aboveskip=1em
}


\pagestyle{fancy}





% Edit these as appropriate
\newcommand\course{差分方法\uppercase\expandafter{\romannumeral2}}
\newcommand\hwnumber{}                  % <-- homework number
\newcommand\name{陈伟}                 % <-- Name
\newcommand\ID{1901110037}           % <-- ID

\pagestyle{fancyplain}
\headheight 35pt
\lhead{\name\\\ID}                 
\chead{\textbf{\Large  差分方法\uppercase\expandafter{\romannumeral2} lab1\hwnumber}}
\rhead{\course \\ \today}
\lfoot{}
\cfoot{}
\rfoot{\small\thepage}
\headsep 1.5em




\title{差分方法\uppercase\expandafter{\romannumeral2}\\
lab1\\
Wide stencil finite difference method 算例\\
陈伟\\
1901110037}

\begin{document}

\maketitle % —— 显示标题
\tableofcontents %—— 制作目录(目录是根据标题自动生成的)

\newpage
\section{问题描述}
用宽模版的有限差分法求解2D的Monge-Ampère方程:
$$\left\{\begin{aligned}
\operatorname{det} D^{2} u &=f \quad \text { in } \Omega \subset \mathbb{R}^{2} \\
u &=g \quad \text { on } \partial \Omega
\end{aligned}\right.$$
其中计算区域为单位正方形, 也即是$\Omega=(0,1)^2.$令$\bm{x}=(x,y)^T,\bm{x}_0=(0.5,05)$. 对于如下三个例子进行数值实验:
\begin{itemize}
\item Smooth and radial example:
$$u(\bm{x})=\text{exp}(\vert{\bm{x}}\vert^2/2),\quad f(\bm{x})=(1+\vert{\bm{x}})\vert^2\text{exp}(\vert{\bm{x}}\vert^2).$$
\item $C^1$ example:
$$u(\boldsymbol{x})=\frac{1}{2}\left(\left(\left|\boldsymbol{x}-\boldsymbol{x}_{0}\right|-0.2\right)^{+}\right)^{2}, \quad f(\boldsymbol{x})=\left(1-\frac{0.2}{\left|\boldsymbol{x}-\boldsymbol{x}_{0}\right|}\right)^{+}.$$
\item Twice differentiable in the interior domain, but has unbounded gradient near the boundary point $(1,1)$:
$$u(\boldsymbol{x})=-\sqrt{2-|\boldsymbol{x}|^{2}}, \quad f(\boldsymbol{x})=2\left(2-|\boldsymbol{x}|^{2}\right)^{-2}.$$
\end{itemize}
边界$g(\bm{x})$可以通过真解获得. 分别用explicit solution method和Newton’s method 来解这个离散的非线性方程. 报告中应包含不同模版的$L^\infty{}$误差.

\section{求解方法}
\subsection{Wide stencil finite difference method}
令$$\operatorname{MA} [\varphi] \left(x_{0}\right)=\min _{\left(w_{1}, \cdots, w_{d}\right) \in V}\left[\prod_{i=1}^{d}\left(\frac{\partial^{2} \varphi}{\partial w_{i}^{2}}\left(x_{0}\right)\right)^{+}-\sum_{i=1}^{d}\left(\frac{\partial^{2} \varphi}{\partial w_{i}^{2}}\left(x_{0}\right)\right)^{-}\right]$$
在$\varphi$是凸函数下有:
$$\operatorname{MA}[\varphi]=\operatorname{det}D^2\varphi$$
对于凸区域$\Omega$,以及网格剖分,给定$\bm{x}_h\in \Omega,$以及方向$\bm{e}$,有$\rho_\pm\in(0,1]$使得$\bm{x}_h\pm \rho_\pm h\bm{e}\in\partial\Omega\cup\bar{\Omega}_h$. 当$\bm{x}_h\pm  h\bm{e}$落在$\Omega_h$内部网格点时,$\rho_\pm$均为1; 当落在外边界时,$\rho_\pm$为使得$\bm{x}_h\pm \rho_\pm h\bm{e}$收缩回边界的比例.令
\begin{equation}\label{Delta_eu}
\begin{array}{c}
\Delta_e u_{h}\left(x_{h}\right) = \frac{2}{\left(\rho_{+}+\rho_{-}\right)|e|^{2} h^{2}}\left[\frac{u_{h}\left(x_{h}+\rho_{+} h e\right)-u_{h}\left(x_{h}\right)}{\rho_{+}}-\frac{u_{h}\left(x_{h}\right)-u_{h}\left(x_{h}-\rho_{-} he\right)}{\rho_{-}}\right]
\end{array}
\end{equation}

以及$G_\theta\subset{}(\mathbb{Z}^d)^d$(参数$d\theta$)为$\mathbb{R}^d$上的正交基组全体$V$的离散.并定义:
$$\operatorname{MA}_{h , \theta}^{WS}[u_{h}] \left(x_{h}\right) \triangleq \min _{\left(v_{1}, \cdots,v_d\right) \in G_{\theta}} \prod_{i=1}^{d}\left(\Delta_{v_{i}} u_{h}\left(x_{n}\right)\right)^{+}$$
对于Monge-Ampère方程的宽模版法也即是:
$$\left\{\begin{aligned}
\operatorname{MA}_{h, \theta}^{WS} \left[u_{h}\right]\left(x_{h}\right) &=f\left(x_{h}\right) & \forall x_{h} \in \Omega_{n} \\
u_{h}\left(x_{h}\right) &=g\left(x_{h}\right) &  \forall x_{h} \in \partial\Omega_{h}
\end{aligned}\right.$$
对于给定的参数$\delta>0$,将上述$\operatorname{MA}_{h, \theta}^{WS}$换成
$$\operatorname{MA}_{h,\theta,\delta}^{WS}\left[u_{h}\right]\left(x_{h}\right) \triangleq {\min _{\left(v_{1}, \cdots, v_{d}\right) \in G_{\theta}}}^{\delta}\left[ \prod_{i=1}^{d}\left(\Delta_{v_{i}} u_{h}\left(x_{h}\right)\right)^{+, \delta}\right]$$
也即得到正则化版本.
总之,我们要求解方程组$F[u]=f,$其中$F$为$\operatorname{MA}_{h,\theta}^{WS}$或者$\operatorname{MA}_{h,\theta,\delta}^{WS}$, 用以下两种迭代法来求解.

\subsection{Explicit solution method}
迭代格式为
$$u^{n+1}=u^n+dt(F[u]-f)$$
其中$dt\thicksim\mathcal{O}(h^2)$,2D case 的一个选择为:
$$d t=\frac{h^{2}}{2}\left(\max _{x_h}
\max _{\left(v_{1}, v_{2}\right) \in G_{\theta}} \left[\left(\Delta_{v_{1}} u^{n}\left(x_{h}\right)\right)^{+}+\left(\Delta_{v_{2}} u^{n}\left(x_{h}\right)\right)^{+}\right]\right)^{-1}$$

\subsection{Newton's method }
迭代格式为:
$$u^{n+1}=u^n-\alpha v^n$$
其中$0<\alpha<1$,为使得残量$\Vert{F(u^n)-f}\Vert$下降的参数.矫正项
$v^n$为满足:
$$\nabla{}_uF[u^n]v^n=F[u^n]-f$$

\section{算法}
观察以上方法,不难发现主要分为两部分:

1. $F[u]$的建立;

2. 迭代法求解$F[u]=f.$\\
下面我们分两部分来构造上述两部分.

\subsection{构建$F[u]$}
无论$F$为$\operatorname{MA}_{h,\theta}^{WS}$或者$\operatorname{MA}_{h,\theta,\delta}^{WS}$,其中最主要的部分都是:
\begin{itemize}
\item $G_\theta$的构建;
\item 对于给定的方向$e$, 以及$x_h\in \Omega_h$, $\Delta{}_eu_h(x_h)$的构建;
\end{itemize}
然后再根据$\operatorname{MA}_{h,\theta}^{WS}$和$\operatorname{MA}_{h,\theta,\delta}^{WS}$的方式去构建即可.
\subsubsection{$G_\theta$算法}
对于给定的模版宽度WideN($\text{WideN}=1,2,3,\cdots{}$),函数$G_\theta(\text{WideN})$输出模版宽度为WideN的正交基中方向在一象限的方向的集合$\{\bm{e}\in \mathbb{Z}^2:\bm{e}_1,\bm{e}_2=0,1,2,\cdots{},N;\ |\bm{e}|\neq0\}$, 而且一旦$\bm{e}$定下来了,则$\bm{e}^\perp=(-\bm{e}_2,\bm{e}_1)$也就给定了.下面是$G_\theta$的算法.

\begin{algorithm}[H]
%\label{G_theta}
\caption{计算$G_\theta$} 
\begin{algorithmic}[]
\setstretch{1.2}
\STATE 输入 WideN\\

\STATE l = 0;
 
\STATE  \textbf{For} i = 1:N

\STATE \qquad \textbf{For} j = 0:N-1

\STATE \qquad \qquad l = l+1;
\STATE \qquad \qquad k = gcd(i,j);

\STATE \qquad \qquad $G_\theta$(l,1) = i/k, $G_\theta$(l,2) = j/k;

\STATE \qquad \textbf{end}

\STATE \textbf{end}

\STATE 删去$G_\theta$中以一行为一项来看重复的项

\STATE 输出$G_\theta$

\end{algorithmic}
\end{algorithm}


\subsubsection{$\Delta_eu_h$算法}
对于给定的方向$e$以及所有内点值$u_h,$我们要计算对应内点的$\Delta_eu_h$的值. 对于$x_h\in \Omega_h,$我们先找$x_{h,\pm}=x_h\pm he$, 若$x_{h,\pm}\in\Omega_h,$则$\rho_\pm=1$,若有$x\pm\notin\Omega_h,$ 则根据边界比例来缩小$\rho_\pm$,使得$x_{h,\pm}=x_h\pm\rho_\pm h e\in\partial\Omega$.  最后再根据\eqref{Delta_eu},即可得到$\Delta_eu_h$.我们这里给定$u_h$为原来2D内点数据按列拉升之后得到的一维数组,这样我们可以根据$x_{h,\pm}$的是否位于该数组中来判断是否越出边界,下面给出该算法.(由于后面用Netwon法时候可能每个点的方向可能不一样,故这里我们的$e$可能是(1,2)的数组,也可能是(Nu,2)的数组,其中Nu为$u_h$的长度.)

\begin{algorithm}[H]
%\label{G_theta}
\caption{计算$\Delta_eu_h$} 
\begin{algorithmic}[]
\setstretch{1.2}
\STATE 输入$u_h,e,h$\\

\STATE 得到$u_h$的长度Nu

\STATE \textbf{IF} size($e$,1)==1


\STATE\qquad $e$ = repmat($e$,size($u_h$));
 
\STATE  \textbf{For} s=1:Nu

\STATE\qquad $u_{h,s}$对应二维数组的横纵指标$i_s,j_s$; 令$\rho_{s,\pm}=1.$

\STATE \qquad 计算$u_{h,s}$依赖值$u_{h,s,\pm}$的横纵指标$$i_{s,\pm}=i_s\pm \rho_{s,\pm}e(s,1),\qquad j_{s,\pm}=j_s\pm \rho_{s,\pm}e(s,2)$$

\STATE\qquad 根据$i_{s,\pm},j_{s,\pm}$来得到按列拉升后$u_{h,s,\pm}$的位置$s_{\pm}$;

\STATE \qquad 如果$s_\pm$均是$1:Nu$中的某个整数,则令:
$u_{h,\pm,s}=u_{h,s_\pm}$;

\STATE \qquad 否则若$s_+$超标, 则根据坐标值$x_{s_+}=(i_{s,+}h,j_{s,+}h)$以及$\partial\Omega$来修正收缩比例$\rho_+$,使得重新得到的$x_{s_+}\in \partial\Omega$, 再令$u_{h,+,s}=g(x_{s_+})$,若$s_-$超标,同理处理.

\STATE \qquad 计算:
$$\begin{array}{c}
\Delta_{e_s}u_{h,s} = \frac{2}{\left(\rho_{+,s}+\rho_{-,s}\right)|e_s|^{2} h^{2}}\left(\frac{u_{h,+,s}-u_{h,s}}{\rho_{+,s}}-\frac{u_{h,s}-u_{h,-,s}}{\rho_{-,s}}\right)
\end{array}$$

\STATE \textbf{end}

\STATE 输出$\Delta_eu_h$
\end{algorithmic}
\end{algorithm}

这样我们得到了模版$G_\theta$与$\Delta_eu_h$的算法,剩下的再根据$\operatorname{MA}_{h,\theta}^{WS}$和$\operatorname{MA}_{h,\theta,\delta}^{WS}$的方式即可构建得到对应的$F[u]$

\subsection{迭代法求解$F[u]=f$}
Explicit solution method 方法按照对应的格式来即可,我们主要讨论Newton's method.其中困难的一步是$\nabla{}_uF[u]$矩阵的构建. 再进一步,两种方法均会用到的是$\nabla{}_u\Delta{}_eu_h$.对于不同的点$x_h,$对应的$e$方向可能也不同.

\subsubsection{$\nabla{}_u\Delta{}_eu_h$算法}
对于给定的点$x_h,$我们求$\nabla{}_u\Delta{}_eu_h(x_h)$,而由\eqref{Delta_eu},我们需要知道对应的$\rho_\pm,$以及$x_h\pm\rho_\pm he$的位置.接算法2,我们有:


\begin{algorithm}[H]
%\label{G_theta}
\caption{计算$\nabla{}_u\Delta_eu_h$} 
\begin{algorithmic}[]
\setstretch{1.2}
\STATE 输入$u_h,e,h$\\

\STATE 得到$u_h$的长度Nu

\STATE \textbf{IF} size($e$,1)==1


\STATE\qquad $e$ = repmat($e$,size($u_h$));
 
\STATE  \textbf{For} s=1:Nu

\STATE\qquad $u_{h,s}$对应二维数组的横纵指标$i_s,j_s$; 令$\rho_{s,\pm}=1.$

\STATE \qquad 计算$u_{h,s}$依赖值$u_{h,s,\pm}$的横纵指标$$i_{s,\pm}=i_s\pm \rho_{s,\pm}e(s,1),\qquad j_{s,\pm}=j_s\pm \rho_{s,\pm}e(s,2)$$

\STATE\qquad 根据$i_{s,\pm},j_{s,\pm}$来得到按列拉升后$u_{h,s,\pm}$的位置$s_{\pm}$;

\STATE \qquad 如果$s_\pm$均是$1:Nu$中的某个整数,则令:
$u_{h,\pm,s}=u_{h,s_\pm}$;

\STATE \qquad 否则若$s_+$超标, 则根据坐标值$x_{s_+}=(i_{s,+}h,j_{s,+}h)$以及$\partial\Omega$来修正收缩比例$\rho_+$,使得重新得到的$x_{s_+}\in \partial\Omega$, 再令$u_{h,+,s}=g(x_{s_+})$,若$s_-$超标,同理处理.

\STATE \qquad 计算:
$$\begin{array}{c}
\nabla{}_u\Delta_{e}u_{h}(s,s) = -\frac{2}{\left(\rho_{+,s}+\rho_{-,s}\right)|e_s|^{2} h^{2}}\left(\frac{1}{\rho_{+,s}}+\frac{1}{\rho_{-,s}}\right)
\end{array}$$

\STATE \qquad 若$1\leqslant s_+\leqslant Nu,$则
$$\begin{array}{c}
\nabla{}_u\Delta_{e}u_{h}(s,s_+) = -\frac{2}{\left(\rho_{+,s}+\rho_{-,s}\right)|e_s|^{2} h^{2}}\frac{1}{\rho_{+,s}}
\end{array}$$

\STATE \qquad 若$1\leqslant s_-\leqslant Nu,$则
$$\begin{array}{c}
\nabla{}_u\Delta_{e}u_{h}(s,s_-) = -\frac{2}{\left(\rho_{+,s}+\rho_{-,s}\right)|e_s|^{2} h^{2}}\frac{1}{\rho_{-,s}}
\end{array}$$

\STATE \textbf{end}

\STATE 输出$\nabla{}_u\Delta_eu_h$
\end{algorithmic}
\end{algorithm}

这样我们就可以分别得到$\operatorname{MA}_{h,\theta}^{WS}$和$\operatorname{MA}_{h,\theta,\delta}^{WS}$的梯度矩阵

\subsubsection{$\nabla{}_u\operatorname{MA}_{h,\theta}^{WS}[u_h]$算法}
用函数乘积的求导法则我们有:
$$\nabla{}_u\operatorname{MA}_{h,\theta}^{WS}[u_h]=\sum_{i=1}^d\left[\textbf{diag}\left(\prod_{s\neq i}\Delta{}_{v^*_s}u_h\right)\nabla{}_u\Delta{}_{v^*_i}u_h\right]$$


\begin{algorithm}[H]
%\label{G_theta}
\caption{计算$\nabla{}_u\operatorname{MA}_{h,\theta}^{WS}[u_h]$} 
\begin{algorithmic}[]
\setstretch{1.2}
\STATE 输入$u_h,G_\theta,h$\\

\STATE 令$\nabla{}_u\operatorname{MA}_{h,\theta}^{WS}[u_h]=0$.

\STATE 在计算$\nabla{}_u\operatorname{MA}_{h,\theta}^{WS}[u_h]$同时得到每个点的最小正交基$v^*=(v^*_1,v^*_2,\cdots{},v^*_d)$.

\STATE \textbf{For} i = 1:d

\STATE  \qquad 调用算法3,输入$u_h,v^*_i,h,$得到对应的$\nabla{}_u\Delta{}_{v^*_i}u_h$

\STATE  \qquad 更新
$$\nabla{}_u\operatorname{MA}_{h,\theta}^{WS}[u_h]=\nabla{}_u\operatorname{MA}_{h,\theta}^{WS}[u_h]+\textbf{diag}\left(\prod_{s\neq i}\Delta{}_{v^*_s}u_h\right)\nabla{}_u\Delta{}_{v^*_i}u_h$$

\STATE  \textbf{end}

\STATE 输出$\nabla{}_u\operatorname{MA}_{h,\theta}^{WS}[u_h]$
\end{algorithmic}
\end{algorithm}

这样我们就得到$\nabla{}_u\operatorname{MA}_{h,\theta}^{WS}[u_h]$.


\subsubsection{$\nabla{}_u\operatorname{MA}_{h,\theta,\delta}^{WS}[u_h]$算法}
由链式法则以及乘积的求导法则以及$\operatorname{MA}_{h,\theta,\delta}^{WS}[u_h]$的表达式,我们有:
$$\nabla{}_u\operatorname{MA}_{h,\theta,\delta}^{WS}[u_h]=\partial_x{\min}^{\delta}(x,y)\nabla{}_u x+\partial_y{\min}^{\delta}(x,y)\nabla{}_u y$$
其中
$$\begin{aligned}
x &= \prod_{i=1}^{d}\left(\Delta_{v_{i}} u_{h}\right)^{+, \delta}\\
y &= \min _{\left(\hat{v}_{1}, \cdots, \hat{v}_{d}\right) \in G_{\theta}\setminus (v_1,\cdots{},v_d)} \delta\left[\prod_{i=1}^{d}\left(\Delta_{v_{i}} u_{h}\right)^{+, \delta}\right]
\end{aligned}$$

不难发现$\nabla{}_uy$可以递归计算. 所以主要是$\nabla{}_ux$的计算.有:
$$\nabla_ux=\sum_{i=1}^n\left[\textbf{diag}\left(\prod_{s=\neq i}(\Delta{}_{v_s}u_h)^{+,\delta}\right)\nabla{}_u(\Delta{}_{v_i}u_h)^{+,\delta}\right]$$

而关于$\nabla{}_u(\Delta{}_{v_i}u_h)^{+,\delta}$,有
$$\begin{aligned}
\nabla{}_u(\Delta{}_{v_i}u_h)^{+,\delta}=\frac{1}{2}\textbf{diag}\left(1+\frac{\Delta{}_{v_i}u_h}{\sqrt{(\Delta{}_{v_i}u_h})^2+\delta^2}\right)\nabla{}_u(\Delta{}_{v_i}u_h)
\end{aligned}$$

这样我们就得到了$\nabla{}_u\operatorname{MA}_{h,\theta,\delta}^{WS}[u_h]$的算法.

\begin{algorithm}[H]
%\label{G_theta}
\caption{计算$\nabla{}_u\operatorname{MA}_{h,\theta,\delta}^{WS}[u_h]$} 
\begin{algorithmic}[]
\setstretch{1.2}
\STATE 输入$u_h,G_\theta,h$\\

\STATE 得到$G_\theta$中向量个数L
\STATE 令$\nabla{}_u\operatorname{MA}_{h,\theta,\delta}^{WS}[u_h]=0$, 令$\nabla{}_u\operatorname{MA}_{h,\theta,\delta}^{WS}[u_h]=0$.

\STATE \textbf{For} l = 1:L

\STATE \qquad \textbf{IF} l==1

\STATE \qquad \qquad 计算$\nabla{}_u\left(\prod_{i=1}^d(\nabla{}_{v_i^l}u_h)^{+,\delta}\right)$并赋值给$\nabla{}_u\operatorname{MA}_{h,\theta,\delta}^{WS}[u_h]$

\STATE\qquad \qquad 计算$\prod_{i=1}^d(\nabla{}_{v_i^l}u_h)^{+,\delta}$,并赋值给$\nabla{}_u\operatorname{MA}_{h,\theta,\delta}^{WS}[u_h]$.

\STATE \qquad \textbf{end}

\STATE \qquad 计算$\nabla{}_u\left(\prod_{i=1}^d(\nabla{}_{v_i^l}u_h)^{+,\delta}\right)$与$\prod_{i=1}^d(\nabla{}_{v_i^l}u_h)^{+,\delta}$,

\STATE \qquad  更新
$$\begin{aligned}
\nabla{}_u\operatorname{MA}_{h,\theta,\delta}^{WS}[u_h]&=\frac{1}{2}\textbf{diag}\left(1-\frac{\operatorname{MA}_{h,\theta,\delta}^{WS}[u_h]-\prod_{i=1}^d(\nabla{}_{v_i^l}u_h)^{+,\delta}}{\sqrt{(\operatorname{MA}_{h,\theta,\delta}^{WS}[u_h])^2+(\prod_{i=1}^d(\nabla{}_{v_i^l}u_h)^{+,\delta})^2+\delta^2}}\right)\nabla{}_u\operatorname{MA}_{h,\theta,\delta}^{WS}[u_h]\\
&+\frac{1}{2}\textbf{diag}\left(1-\frac{\prod_{i=1}^d(\nabla{}_{v_i^l}u_h)^{+,\delta}-\operatorname{MA}_{h,\theta,\delta}^{WS}[u_h]}{\sqrt{(\operatorname{MA}_{h,\theta,\delta}^{WS}[u_h])^2+(\prod_{i=1}^d(\nabla{}_{v_i^l}u_h)^{+,\delta})^2+\delta^2}}\right)\nabla{}_u\left[\prod_{i=1}^d(\nabla{}_{v_i^l}u_h)^{+,\delta}\right]
\end{aligned}$$

\STATE \qquad 更新
$$\operatorname{MA}_{h,\theta,\delta}^{WS}[u_h]={\min}^{\delta}\left(\prod_{i=1}^d(\nabla{}_{v_i^l}u_h)^{+,\delta},\operatorname{MA}_{h,\theta,\delta}^{WS}[u_h]\right)$$

\STATE  \textbf{end}

\STATE 输出$\nabla{}_u\operatorname{MA}_{h,\theta,\delta}^{WS}[u_h]$
\end{algorithmic}
\end{algorithm}

这样,程序的主要部分就完成了. 后面部分是一些数值算例.

\section{数值算例}

下面对三个例子分别用Explicit solution method和Newton's method, 选取函数$u_0(x,y)=x^2+y^2$的值为迭代初值,取迭代终止条件为
$$\Vert{u^{n+1}-u^n}\Vert_\infty{}\leqslant 1e-8\text{  or  迭代次数} \geqslant 5000$$
得到的结果如下.

取$h = 1/8,1/16,1/32$,模版宽度为$1,2,3,4$.

关于非正则的Explicit solution method, 对于Examlpe1,很遗憾,达迭代上限后也没达收敛精度. 重新随机选取初值,结果也是如此。总之,需要很大的迭代步此方法才可能收敛.

对于正则的Explicit solution method,分别取$\delta=10,1,10^{-2},$结果如下

\begin{table}[ht!]
\centering
\caption{正则化Explicit solution method}
\label{table}
    \begin{tabular}{  c| c|c c|c c| c c }
          \hline      
  \hline
  
\hline



    \hline     
    \hline
    \multicolumn{8}{c}{example1}\\
         \hline      
  \hline
  
\hline



    \hline     
    \hline   
    \multirow{6}{*}{$\delta=10$}
   &wide$\setminus$  h:	&\multicolumn{2}{|c|}{1/8}&\multicolumn{2}{|c|}{1/16}&\multicolumn{2}{|c}{1/32}\\
&	&迭代次数	&误差	&迭代次数	&误差  &迭代次数	&误差		\\
&1&1495	&1.18e+00	&5000	&1.18e+00	&5000	&9.97e-01\\
&2&665	&7.39e-01	&3600	&7.31e-01	&5000	&7.05e-01\\
&3&390	&4.26e-01	&1965	&4.18e-01	&5000	&4.14e-01\\
&4&304	&2.65e-01	&1459	&2.57e-01	&5000	&2.55e-01\\
\iffalse
    \cline{1-8}     	
      \multirow{6}{*}{$\delta=1$}
   &wide$\setminus$  h:	&\multicolumn{2}{|c|}{1/8}&\multicolumn{2}{|c|}{1/16}&\multicolumn{2}{|c}{1/32}\\ 
&	&迭代次数	&误差	&迭代次数	&误差  &迭代次数	&误差		\\ 

&1&467	&7.01e-03	&2171	&5.60e-03	&5000	&1.91e-02\\
&2&453	&1.89e-02	&1998	&2.03e-02	&5000	&2.86e-02\\
&3&457	&4.35e-02	&1936	&4.63e-02	&5000	&5.43e-02\\
&4&488	&6.00e-02	&1923	&6.38e-02	&5000	&7.22e-02\\

\fi

    \cline{1-8}     	
      \multirow{6}{*}{$\delta=10^{-2}$}
   &wide$\setminus$  h:	&\multicolumn{2}{|c|}{1/8}&\multicolumn{2}{|c|}{1/16}&\multicolumn{2}{|c}{1/32}\\ 
&	&迭代次数	&误差	&迭代次数	&误差  &迭代次数	&误差		\\ 

&1&476	&3.55e-03	&2253	&2.01e-03	&5000	&1.60e-01\\
&2&485	&1.69e-03	&2197	&9.94e-04	&5000	&4.90e-02\\
&3&497	&7.56e-03	&2264	&1.43e-03	&5000	&6.41e-02\\
&4&536	&1.52e-02	&2316	&3.29e-03	&5000	&7.57e-02\\

      \hline      
  \hline
  
\hline



    \hline     
    \hline
    \multicolumn{8}{c}{example2}\\
         \hline      
  \hline
  
\hline



    \hline     
    \hline
    \multirow{6}{*}{$\delta=10$}
   &wide$\setminus$  h:	&\multicolumn{2}{|c|}{1/8}&\multicolumn{2}{|c|}{1/16}&\multicolumn{2}{|c}{1/32}\\
&	&迭代次数	&误差	&迭代次数	&误差  &迭代次数	&误差		\\
&1&1105	&1.38e+00	&5000	&1.38e+00	&5000	&1.65e+00\\
&2&388	&7.20e-01	&3201	&7.14e-01	&5000	&9.38e-01\\
&3&171	&3.33e-01	&1405	&3.28e-01	&5000	&4.60e-01\\
&4&102	&1.47e-01	&838	&1.42e-01	&5000	&1.42e-01\\
    \cline{1-8}     	
      \multirow{6}{*}{$\delta=10^{-2}$}
   &wide$\setminus$  h:	&\multicolumn{2}{|c|}{1/8}&\multicolumn{2}{|c|}{1/16}&\multicolumn{2}{|c}{1/32}\\ 
&	&迭代次数	&误差	&迭代次数	&误差  &迭代次数	&误差		\\ 

&1&5000	&8.43e-01	&5000	&1.41e+00	&5000	&1.72e+00\\
&2&5000	&7.89e-01	&5000	&1.37e+00	&5000	&1.71e+00\\
&3&5000	&7.51e-01	&5000	&1.38e+00	&5000	&1.71e+00\\
&4&5000	&7.43e-01	&5000	&1.37e+00	&5000	&1.71e+00\\

      \hline      
  \hline
  
\hline




    \hline     
    \hline
    \multicolumn{8}{c}{example3}\\
         \hline      
  \hline
  
\hline



    \hline     
    \hline
    \multirow{6}{*}{$\delta=10$}
   &wide$\setminus$  h:	&\multicolumn{2}{|c|}{1/8}&\multicolumn{2}{|c|}{1/16}&\multicolumn{2}{|c}{1/32}\\
&	&迭代次数	&误差	&迭代次数	&误差  &迭代次数	&误差		\\
&1&4196	&1.32e+00	&5000	&2.44e+00	&5000	&2.27e+00\\
&2&1724	&7.55e-01	&5000	&1.74e+00	&5000	&2.18e+00\\
&3&963	&3.97e-01	&5000	&7.40e-01	&5000	&2.07e+00\\
&4&724	&2.20e-01	&5000	&2.56e-01	&5000	&1.99e+00\\
 
      \hline      
  \hline
  
\hline


    \end{tabular}
\end{table}
可以看出,$h$越小,解越不光滑, 则对$MA^{WS}_{h,\theta,\delta}$的正则要求越高,即是需要$\delta$越大. 在第三个例子中取$\delta=10$也不能在$5000$步内收敛,同时$\delta$越大,收敛得到的数值解与真解的误差也就越大.总的来说效果不理想。下面我们考虑Netwon法.



由于非正则化的Netwon法得到的梯度矩阵奇异性较大,初值只有落在真解的一个小区间内才收敛。故我们测试正则化的Netwon法,同时对应的正则化因子$\delta$大时,梯度矩阵奇异性较小,但数值解与真解的误差较大,故我们可以先选取较大的$\delta$,等用Netwon法收敛之后,减小$\delta$,再用之前得到的数值解作为迭代初值,再用正则化的Netwon法迭代至解收敛,再减小$\delta$(如令$\delta^{n+1}=\delta^{n}/2$),如此重复,直到$\delta$充分小或者本次迭代收敛值和初值相差充分小即可。对于三个数值算例, 分别取定$\delta_{max},$取收敛条件为:
$$\delta^{n}<1e-6\text{ or  }\Vert{u^{n}-u^{n-1}}\Vert_\infty{}<1e-8$$
其中$u^{n}$为以$u^{n-1}$为初值,$\delta$取为$\delta^n$的收敛结果.

\begin{table}[ht!]
\centering
\caption{正则化Explicit solution method}
\label{table}
    \begin{tabular}{  c| c|c c|c c| c c }
          \hline      
  \hline
  
\hline



    \hline     
    \hline
    \multicolumn{8}{c}{example1}\\
         \hline      
  \hline
  
\hline



    \hline     
    \hline   
    \multirow{6}{*}{$\delta_{max}=2$}
   &wide$\setminus$  h:	&\multicolumn{2}{|c|}{1/16}&\multicolumn{2}{|c|}{1/32}&\multicolumn{2}{|c}{1/64}\\
&	&迭代次数	&误差	&迭代次数	&误差  &迭代次数	&误差		\\
&1&171	&2.03e-03	&177	&1.69e-03	&180	&1.61e-03\\
&2&179	&1.07e-03	&186	&6.37e-04	&190	&4.92e-04\\
&3&184	&1.42e-03	&192	&4.27e-04	&202	&2.51e-04\\
&4&183	&3.27e-03	&199	&6.02e-04	&217	&2.09e-04\\      \hline      
  \hline
  
\hline



    \hline     
    \hline
    \multicolumn{8}{c}{example2}\\
         \hline      
  \hline
  
\hline



    \hline     
    \hline
    \multirow{6}{*}{$\delta_{max}=5$}
   &wide$\setminus$  h:	&\multicolumn{2}{|c|}{1/8}&\multicolumn{2}{|c|}{1/16}&\multicolumn{2}{|c}{1/32}\\
&	&迭代次数	&误差	&迭代次数	&误差  &迭代次数	&误差		\\
&1&1092	&2.59e-03	&558	&2.38e-03	&1150	&2.13e-03\\
&2&487	&1.71e-03	&384	&8.02e-04	&466	&6.63e-04\\
&3&370	&1.15e-03	&334	&5.65e-04	&348	&3.12e-04\\
&4&340	&1.84e-03	&338	&4.53e-04	&368	&2.42e-04\\
        \hline      
  \hline
  
\hline




    \hline     
    \hline
    \multicolumn{8}{c}{example3}\\
         \hline      
  \hline
  
\hline



    \hline     
    \hline
    \multirow{6}{*}{$\delta=5$}
   &wide$\setminus$  h:	&\multicolumn{2}{|c|}{1/8}&\multicolumn{2}{|c|}{1/16}&\multicolumn{2}{|c}{1/32}\\
&	&迭代次数	&误差	&迭代次数	&误差  &迭代次数	&误差		\\
&1&315	&4.45e-03	&486	&1.58e-03	&10	&4.67e+06\\
&2&240	&4.40e-03	&273	&1.58e-03	&25	&1.79e+07\\
&3&225	&1.97e-03	&258	&1.03e-03	&20	&4.80e+05\\
&4&226	&3.99e-03	&257	&1.46e-03	&42	&2.28e+06\\
  \cline{1-8}     	
      \multirow{6}{*}{$\delta=10$}
   &wide$\setminus$  h:	&\multicolumn{2}{|c|}{1/8}&\multicolumn{2}{|c|}{1/16}&\multicolumn{2}{|c}{1/32}\\ 
&	&迭代次数	&误差	&迭代次数	&误差  &迭代次数	&误差		\\ 

&1&492	&4.45e-03	&778	&1.58e-03	&1561	&8.62e-04\\
&2&314	&4.40e-03	&416	&1.58e-03	&578	&5.59e-04\\
&3&271	&1.97e-03	&313	&1.03e-03	&351	&5.59e-04\\
&4&261	&3.99e-03	&302	&1.46e-03	&360	&3.45e-04\\      \hline      
  \hline
  
\hline


    \end{tabular}
\end{table}
可以看出,Netwon法的收敛速度比显示法快很多。但是Netwon法得到的矩阵可能有较强的奇异性,只有在初值选取较好时才能较好收敛,为此好的初值选取很重要。
\section{代码说明}
本次程序使用matlab编写,包含main.m和Readme.txt等在内一共有24个文件,在文件夹‘matlabcode’中. 关于各文件的含义以及main.m的运行参见Readme.txt.





\end{document}